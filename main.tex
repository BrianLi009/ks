\documentclass[a4paper]{article}

\usepackage{draftwatermark}
\usepackage{paralist}
\usepackage{authblk}
\usepackage{amsthm}

% TODO remove this in the final version
\newcommand{\keyword}[1]{\textbf{#1}}

\newcounter{main}
\newtheorem{prop}[main]{Proposition}
\newtheorem{thm}[main]{Theorem}
\newtheorem{lem}[main]{Lemma}
\theoremstyle{definition}
\newtheorem{dfn}[main]{Definition}
\newtheorem*{spin}{SPIN Axiom \cite{ck09}}
\theoremstyle{remark}
\newtheorem{rem}[main]{Remark}

% TODO this suggests only one result: the improved lower bound
%      reference to other results?
\title{A Kochen-Specker system has at least 22 vertices}

\author{Sander Uijlen}
\author{Bas Westerbaan}

% TODO will we keep these @cs.ru.nl addresses?
\affil{Institute for Computing and Information Sciences\\
       Radboud Universiteit Nijmegen\\
   \{\texttt{suijlen},\texttt{bwesterb}\}\texttt{@cs.ru.nl}}

\begin{document}

\maketitle

\begin{abstract}
    At the heart of the Conway's Free Will theorems and Kochen and Specker's
        argument against noncontextual hidden variable theories
    is the existence of a Kochen-Specker (KS) system:
    a set of points on the sphere,
    that has no~$\{0,1\}$-coloring such that
    at most one of two orthogonal points are colored~$1$
    and of three pairwise orthogonal points exactly one
    is colored~$1$.
    In public lectures, Conway encouraged the search for small
    KS systems.  
    At the time of writing, the smallest known
    KS system has 31 vectors.  

    Arends, Ouaknine and Wampler have shown that a KS system has at least
    18 vectors, by reducing the problem to the existence of graphs
    with a topological embeddability and non-colorability property.
    Deciding embeddability and the sheer number of graphs on more than~$17$
    vertices, proved the bottleneck in their search.

    Continuing their effort, we restrict our enumeration to smaller class of
    graphs and develop a practical decision procedure for embeddability, to
    improve the lower bound to 22.
\end{abstract}
    
\section{Introduction}

% TODO whose idea is this orthogonality graph
\subsection{The experiment}

% TODO correct description of experiment
Let's perform the following experiment.  Shoot a deuterium atom,
or any other spin-1 particle,
along, say: the x-axis, through a uniform magnetic field.
Depending on the direction of the magnetic field,
the particle will move undisturbed
or deviate.

Quantum Mechanics only predicts the chance, given the initial configuration
of the field, whether the particle will deviate.
Its probabilistic prediction has been thouroughly tested.
One wonders: is there a deterministic theory predicting the
outcome of this experiment?

Kochen and Specker proved that such a theory cannot satisfy:
% TODO note contextuality
\begin{spin}
    Given three pairwise orthogonal directions.
    In exactly one of the directions, the particle will not deviate.
\end{spin}
Their argument is based on the existence of a Kochen-Specker system.
\begin{dfn}
    A \keyword{Kochen-Specker (KS) system} is
    a finite set of points on the sphere
    for which each pair is not antipodal and
    there is no~\keyword{010-coloring}.
    A $010$-coloring is a~$\{0,1\}$-coloring of the points such that
    \begin{enumerate}
        \item
            no pair of orthogonal points are both colored~$1$ and
        \item
            of three pairwise orthogonal points exactly one is colored~$1$;
            or alternatively: they are colored~$0$, $1$ and~$0$ in some order.
    \end{enumerate}
\end{dfn}
% TODO shorten this proof.
For suppose there is a KS system and  a deterministic theory satisfying
the SPIN Axiom.
Then we color a point of the system~$0$,
whenever the theory predicts that the particle will deviate
if the magnetic field is directed along that point and~$1$ otherwise.
Given two orthogonal points of the system,
we can find a third point orthogonal to both of them.
The SPIN axiom implies exactly one of them is colored~$1$.
Thus they cannot both be colored~$1$.
And similarly, given three pairwise orthogonal vectors in the system,
the SPIN axiom implies exactly one of them is colored~$1$.
Hence the KS system is 010-colorable, quod non.  Thus the deterministic theory
cannot satisfy the SPIN Axiom.

The KS system proposed by Kochen and Specker contained 117 points\cite{ks}.
% TODO reference (x2)
Penrose and Peres independently found a smaller system of 33 points.
The current record is the 31 point system of Conway.
As pointed out by \cite{c00,aow11}, finding small KS systems
is of both theoretical and practical interest.
In public lectures, Conway himself, stressed the search for small KS systems.

% TODO refer to non-3d systems
In \cite{aow11} Arends, Ouaknine and Wampler show that a KS system must have
at least 18 vectors.  We have improved upon their result, abd

\subsection{The 33 vector KS system of Penrose and Peres}


\section{An improved lower bound}
TODO

\section{Embeddability}
TODO

\section{Acknowledgments}
We wish to thank the following for their generous contribution to the
distributed computation:
    the Digital Security group, Intelligent Systems group
    and the C\&CZ service of the Radboud University;
    Wouter Geraedts and
    Jille Timmermans.

% attribute mckay for answering questions?
\bibliography{main}{}
\bibliographystyle{plain}


\end{document}

% vim: ft=tex.latex
